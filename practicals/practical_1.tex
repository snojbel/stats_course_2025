% Options for packages loaded elsewhere
\PassOptionsToPackage{unicode}{hyperref}
\PassOptionsToPackage{hyphens}{url}
%
\documentclass[
]{article}
\usepackage{amsmath,amssymb}
\usepackage{iftex}
\ifPDFTeX
  \usepackage[T1]{fontenc}
  \usepackage[utf8]{inputenc}
  \usepackage{textcomp} % provide euro and other symbols
\else % if luatex or xetex
  \usepackage{unicode-math} % this also loads fontspec
  \defaultfontfeatures{Scale=MatchLowercase}
  \defaultfontfeatures[\rmfamily]{Ligatures=TeX,Scale=1}
\fi
\usepackage{lmodern}
\ifPDFTeX\else
  % xetex/luatex font selection
\fi
% Use upquote if available, for straight quotes in verbatim environments
\IfFileExists{upquote.sty}{\usepackage{upquote}}{}
\IfFileExists{microtype.sty}{% use microtype if available
  \usepackage[]{microtype}
  \UseMicrotypeSet[protrusion]{basicmath} % disable protrusion for tt fonts
}{}
\makeatletter
\@ifundefined{KOMAClassName}{% if non-KOMA class
  \IfFileExists{parskip.sty}{%
    \usepackage{parskip}
  }{% else
    \setlength{\parindent}{0pt}
    \setlength{\parskip}{6pt plus 2pt minus 1pt}}
}{% if KOMA class
  \KOMAoptions{parskip=half}}
\makeatother
\usepackage{xcolor}
\usepackage[margin=1in]{geometry}
\usepackage{color}
\usepackage{fancyvrb}
\newcommand{\VerbBar}{|}
\newcommand{\VERB}{\Verb[commandchars=\\\{\}]}
\DefineVerbatimEnvironment{Highlighting}{Verbatim}{commandchars=\\\{\}}
% Add ',fontsize=\small' for more characters per line
\usepackage{framed}
\definecolor{shadecolor}{RGB}{248,248,248}
\newenvironment{Shaded}{\begin{snugshade}}{\end{snugshade}}
\newcommand{\AlertTok}[1]{\textcolor[rgb]{0.94,0.16,0.16}{#1}}
\newcommand{\AnnotationTok}[1]{\textcolor[rgb]{0.56,0.35,0.01}{\textbf{\textit{#1}}}}
\newcommand{\AttributeTok}[1]{\textcolor[rgb]{0.13,0.29,0.53}{#1}}
\newcommand{\BaseNTok}[1]{\textcolor[rgb]{0.00,0.00,0.81}{#1}}
\newcommand{\BuiltInTok}[1]{#1}
\newcommand{\CharTok}[1]{\textcolor[rgb]{0.31,0.60,0.02}{#1}}
\newcommand{\CommentTok}[1]{\textcolor[rgb]{0.56,0.35,0.01}{\textit{#1}}}
\newcommand{\CommentVarTok}[1]{\textcolor[rgb]{0.56,0.35,0.01}{\textbf{\textit{#1}}}}
\newcommand{\ConstantTok}[1]{\textcolor[rgb]{0.56,0.35,0.01}{#1}}
\newcommand{\ControlFlowTok}[1]{\textcolor[rgb]{0.13,0.29,0.53}{\textbf{#1}}}
\newcommand{\DataTypeTok}[1]{\textcolor[rgb]{0.13,0.29,0.53}{#1}}
\newcommand{\DecValTok}[1]{\textcolor[rgb]{0.00,0.00,0.81}{#1}}
\newcommand{\DocumentationTok}[1]{\textcolor[rgb]{0.56,0.35,0.01}{\textbf{\textit{#1}}}}
\newcommand{\ErrorTok}[1]{\textcolor[rgb]{0.64,0.00,0.00}{\textbf{#1}}}
\newcommand{\ExtensionTok}[1]{#1}
\newcommand{\FloatTok}[1]{\textcolor[rgb]{0.00,0.00,0.81}{#1}}
\newcommand{\FunctionTok}[1]{\textcolor[rgb]{0.13,0.29,0.53}{\textbf{#1}}}
\newcommand{\ImportTok}[1]{#1}
\newcommand{\InformationTok}[1]{\textcolor[rgb]{0.56,0.35,0.01}{\textbf{\textit{#1}}}}
\newcommand{\KeywordTok}[1]{\textcolor[rgb]{0.13,0.29,0.53}{\textbf{#1}}}
\newcommand{\NormalTok}[1]{#1}
\newcommand{\OperatorTok}[1]{\textcolor[rgb]{0.81,0.36,0.00}{\textbf{#1}}}
\newcommand{\OtherTok}[1]{\textcolor[rgb]{0.56,0.35,0.01}{#1}}
\newcommand{\PreprocessorTok}[1]{\textcolor[rgb]{0.56,0.35,0.01}{\textit{#1}}}
\newcommand{\RegionMarkerTok}[1]{#1}
\newcommand{\SpecialCharTok}[1]{\textcolor[rgb]{0.81,0.36,0.00}{\textbf{#1}}}
\newcommand{\SpecialStringTok}[1]{\textcolor[rgb]{0.31,0.60,0.02}{#1}}
\newcommand{\StringTok}[1]{\textcolor[rgb]{0.31,0.60,0.02}{#1}}
\newcommand{\VariableTok}[1]{\textcolor[rgb]{0.00,0.00,0.00}{#1}}
\newcommand{\VerbatimStringTok}[1]{\textcolor[rgb]{0.31,0.60,0.02}{#1}}
\newcommand{\WarningTok}[1]{\textcolor[rgb]{0.56,0.35,0.01}{\textbf{\textit{#1}}}}
\usepackage{graphicx}
\makeatletter
\def\maxwidth{\ifdim\Gin@nat@width>\linewidth\linewidth\else\Gin@nat@width\fi}
\def\maxheight{\ifdim\Gin@nat@height>\textheight\textheight\else\Gin@nat@height\fi}
\makeatother
% Scale images if necessary, so that they will not overflow the page
% margins by default, and it is still possible to overwrite the defaults
% using explicit options in \includegraphics[width, height, ...]{}
\setkeys{Gin}{width=\maxwidth,height=\maxheight,keepaspectratio}
% Set default figure placement to htbp
\makeatletter
\def\fps@figure{htbp}
\makeatother
\setlength{\emergencystretch}{3em} % prevent overfull lines
\providecommand{\tightlist}{%
  \setlength{\itemsep}{0pt}\setlength{\parskip}{0pt}}
\setcounter{secnumdepth}{-\maxdimen} % remove section numbering
\ifLuaTeX
  \usepackage{selnolig}  % disable illegal ligatures
\fi
\usepackage{bookmark}
\IfFileExists{xurl.sty}{\usepackage{xurl}}{} % add URL line breaks if available
\urlstyle{same}
\hypersetup{
  pdftitle={Practical 1},
  pdfauthor={Izabel},
  hidelinks,
  pdfcreator={LaTeX via pandoc}}

\title{Practical 1}
\author{Izabel}
\date{2025-01-22}

\begin{document}
\maketitle

I have chosen to do the analysis required for this practical in R, using
the pwr package.

\subsubsection{Question 1.}\label{question-1.}

\emph{Correlation between X and Y. You suspect that a true corrleation
is somewhere around r=0.3. How large sample size (i.e; how many X-Y
pairs) do you need to reach a power of 80\% for an alpha of 5\%?}

\begin{verbatim}
## Warning: package 'pwr' was built under R version 4.4.2
\end{verbatim}

\begin{Shaded}
\begin{Highlighting}[]
\FunctionTok{pwr.r.test}\NormalTok{(}\AttributeTok{n =} \ConstantTok{NULL}\NormalTok{, }\AttributeTok{r =} \FloatTok{0.30}\NormalTok{, }\AttributeTok{sig.level =} \FloatTok{0.05}\NormalTok{, }\AttributeTok{power =} \FloatTok{0.8}\NormalTok{, }\AttributeTok{alternative =} \StringTok{"two.sided"}\NormalTok{)}
\end{Highlighting}
\end{Shaded}

\begin{verbatim}
## 
##      approximate correlation power calculation (arctangh transformation) 
## 
##               n = 84.07364
##               r = 0.3
##       sig.level = 0.05
##           power = 0.8
##     alternative = two.sided
\end{verbatim}

The results suggest that a sample size of 84 is desirable for this type
of test, assuming the actual effect size is as high as 0.3. If the
actual correlation is 0.28, the analysis suggests as many as
\textasciitilde97 samples are needed. So I would aim for a slightly
conservative estimate off the effect size.

\subsubsection{Question 2.}\label{question-2.}

\emph{You are reading a paper where the authors have taken samples from
sites of chemical contamination of lead (Pb) in rivers. They then
correlate the strength of contamination with a measure of species
diversity of aquatic macroinvertebrates across 14 sites. They find no
significant correlation (P \textgreater{} 0.05) and conclude that lead
contamination has no effect on the macroinvertebrate fauna. What do you
think about this conslusion and what is the power to detect a weak
effect (r=0.1) of lead contamination in this study?}

This study has a sample size of 14 and using the effect size of 0.1 we
can calculate the expected power of a 0.05 significant test:

\begin{Shaded}
\begin{Highlighting}[]
\FunctionTok{pwr.r.test}\NormalTok{(}\AttributeTok{n =} \DecValTok{14}\NormalTok{, }\AttributeTok{r =} \FloatTok{0.1}\NormalTok{, }\AttributeTok{sig.level =} \FloatTok{0.05}\NormalTok{, }\AttributeTok{power =} \ConstantTok{NULL}\NormalTok{, }\AttributeTok{alternative =} \StringTok{"two.sided"}\NormalTok{)}
\end{Highlighting}
\end{Shaded}

\begin{verbatim}
## 
##      approximate correlation power calculation (arctangh transformation) 
## 
##               n = 14
##               r = 0.1
##       sig.level = 0.05
##           power = 0.06263932
##     alternative = two.sided
\end{verbatim}

This suggests a power of \textasciitilde6\%, it is very unlikely that
they would actually detect anything. So their results is most likely a
type-II error.

\subsubsection{Question 3.}\label{question-3.}

\emph{You wish to test for a difference between two groups with a
t-test. Preliminary data suggest that one mean is around 4 and the other
is close to 5, and the standard deviation in both groups is about 2. You
set alpha to 5\%. How many observations do you need to do in each group
(the same sample size in both groups; N1=N2) to reach a power of 80\%?}

\begin{Shaded}
\begin{Highlighting}[]
\NormalTok{d }\OtherTok{\textless{}{-}}\NormalTok{ (}\DecValTok{5} \SpecialCharTok{{-}} \DecValTok{4}\NormalTok{)}\SpecialCharTok{/}\DecValTok{2}    \CommentTok{\# Cohen\textquotesingle{}s d, the effect size. Difference between the means divided by the pooled standard deviation}

\FunctionTok{power.t.test}\NormalTok{(}\AttributeTok{n =} \ConstantTok{NULL}\NormalTok{, }\AttributeTok{d =}\NormalTok{ d, }\AttributeTok{sig.level =} \FloatTok{0.05}\NormalTok{, }\AttributeTok{power =} \FloatTok{0.8}\NormalTok{, }\AttributeTok{type =} \StringTok{"two.sample"}\NormalTok{, }\AttributeTok{alternative =} \StringTok{"two.sided"}\NormalTok{)}
\end{Highlighting}
\end{Shaded}

\begin{verbatim}
## 
##      Two-sample t test power calculation 
## 
##               n = 63.76576
##           delta = 0.5
##              sd = 1
##       sig.level = 0.05
##           power = 0.8
##     alternative = two.sided
## 
## NOTE: n is number in *each* group
\end{verbatim}

This suggests that I would need atleast 64 observations from each group
to reach a power of 80\%.

\subsubsection{Question 4.}\label{question-4.}

\emph{You are planning an experiment where you grow plants in three soil
types. A pilot study suggests that plant biomass will be about 23, 25
and 29 in your three treatment groups. Standard deviation seems to be
about 6 in all three groups. You will analyse your results with a
one-way analysis of variance (ANOVA), but need to know how many
replicates you need: what should the sample size be in each group (use
the same sample size in all groups; N1=N2=N3) to get an approximate
power of 80\%?}

\begin{Shaded}
\begin{Highlighting}[]
\CommentTok{\# f is the standard deviation of standardized means}
\NormalTok{pop\_mean }\OtherTok{\textless{}{-}}\NormalTok{ (}\DecValTok{23} \SpecialCharTok{+} \DecValTok{25} \SpecialCharTok{+} \DecValTok{29}\NormalTok{) }\SpecialCharTok{/} \DecValTok{3}
\NormalTok{f }\OtherTok{\textless{}{-}} \FunctionTok{sqrt}\NormalTok{(}\FunctionTok{sum}\NormalTok{(}\FunctionTok{c}\NormalTok{((}\DecValTok{23} \SpecialCharTok{{-}}\NormalTok{ pop\_mean)}\SpecialCharTok{\^{}}\DecValTok{2}\NormalTok{, (}\DecValTok{25} \SpecialCharTok{{-}}\NormalTok{ pop\_mean)}\SpecialCharTok{\^{}}\DecValTok{2}\NormalTok{, (}\DecValTok{29} \SpecialCharTok{{-}}\NormalTok{ pop\_mean)}\SpecialCharTok{\^{}}\DecValTok{2}\NormalTok{))}\SpecialCharTok{/}\DecValTok{3}\NormalTok{)}\SpecialCharTok{/}\DecValTok{6}
\FunctionTok{pwr.anova.test}\NormalTok{(}\AttributeTok{k =} \DecValTok{3}\NormalTok{, }\AttributeTok{n =} \ConstantTok{NULL}\NormalTok{, }\AttributeTok{f =}\NormalTok{ f, }\AttributeTok{sig.level =} \FloatTok{0.05}\NormalTok{, }\AttributeTok{power =} \FloatTok{0.8}\NormalTok{)}
\end{Highlighting}
\end{Shaded}

\begin{verbatim}
## 
##      Balanced one-way analysis of variance power calculation 
## 
##               k = 3
##               n = 19.61508
##               f = 0.4157397
##       sig.level = 0.05
##           power = 0.8
## 
## NOTE: n is number in each group
\end{verbatim}

This suggests that I should plan to get at least 20 samples in each
group. But like mentioned previously, as all calculations of effect size
are estimates and best guesses of actual differences I would either aim
for more samples or choose somewhat conservative estimates when
predicting effect size.

\end{document}
